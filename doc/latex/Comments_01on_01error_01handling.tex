Please go to \hyperlink{error_8finc}{api/error.\-finc} for the A\-P\-I documentation.

The error handling subroutines are available from Fortran, with exception of the macros {\ttfamily G\-S\-L\-\_\-\-E\-R\-R\-O\-R} and {\ttfamily G\-S\-L\-\_\-\-E\-R\-R\-O\-R\-\_\-\-V\-A\-L}. A user-\/defined error handler can be defined either in C or using a Fortran function with the {\ttfamily bind(c)} attribute. Here is the description of the required interface\-: 
\begin{DoxyPre}
 subroutine errhand(reason, file, line, errno) bind(c)
    type(c\_ptr), value :: reason, file
    integer(c\_int), value :: line, errno
 end subroutine errhand
 \end{DoxyPre}
 An object of type {\ttfamily fgsl\-\_\-error\-\_\-handler\-\_\-t} is returned by the constructor {\ttfamily fgsl\-\_\-error\-\_\-handler\-\_\-init(errhand)}, which takes a subroutine with the interface described above as its argument. The subroutine {\ttfamily fgsl\-\_\-error(reason, file, line, errno)} works in an analogous manner as the C version. If the Fortran preprocessor is supported, it should be possible to use the macros {\ttfamily \-\_\-\-\_\-\-F\-I\-L\-E\-\_\-\-\_\-} and {\ttfamily \-\_\-\-\_\-\-L\-I\-N\-E\-\_\-\-\_\-} in the above call. Once not needed any more, the error handler object can be deallocated by calling the subroutine {\ttfamily fgsl\-\_\-error\-\_\-handler\-\_\-free} with itself as its only argument. Note that the function {\ttfamily fgsl\-\_\-strerror} returns a string of length {\ttfamily fgsl\-\_\-strmax}. 